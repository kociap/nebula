\documentclass[12pt, a4paper]{article}

\usepackage[T1]{fontenc}
\usepackage[left=4cm, right=4cm]{geometry}
\usepackage{setspace}

\renewcommand{\labelitemi}{-}

\newenvironment{itemlist}
{\begin{itemize}
  \setlength{\itemsep}{0pt}}
{\end{itemize}}

\onehalfspacing

\title{Architecture and Development Documentation of CODENAME}
\author{Piotr Kocia}
\date{}

\begin{document}
\maketitle

\tableofcontents

\section{Introduction}
The purpose of this document is to explain the architecture and architectural decisions, the
development process and the structure of CODENAME. The aim is to shed light on what CODENAME will be
at the end of the planned development, what features will be present and what risks are possible
throughout the entire process. Additionally, this document steps into some lower level architectural
details of certain systems so as to elucidate why certain decisions have been made.

\section{CODENAME}
CODENAME is a standalone desktop program for designing, building and simulating integrated
electronic circuits (circuits). The end goal is to be able to create circuits from basic logic
building blocks (logic gates, gates) by connecting them with wires, simulate their working by
providing binary inputs and transform the representation into technical diagrams within the program
itself. The program will be run as a standalone desktop window program where the user interacts with
it via user interface elements.

The core planned features are:
\begin{itemlist}
\item realtime rendering of the circuits,
\item realtime interaction and modification of the circuits,
\item grouping of components into larger units for reuse,
\item simulation of execution for given binary inputs,
\item clock-based continuous simulation,
\item persistence of created designs.
\end{itemlist}

Several interesting features are sadly out of the scope of this project at the time due to the
imposed timeframe, hence we consider them extension features which will be implemented once all core
features are completed and time budget allows. Such features are:
\begin{itemlist}
\item automatic layouting,
\item generation of technical diagrams.
\end{itemlist}

\section{Development Phases}
The development is divided into 4 phases: setup, bootstrap, core development, finalisation.

\subsection{Setup}
In the setup phase the goal is to create a formalise the processes within the project. This
encompases initial project setup, standardisation and establishment of guidelines and preparation of
workflows, as well as environment setup, acquisition of tools and brief technical onboarding.

A hard time estimate for this phase is 1 week with no planned overruns.

\subsection{Bootstrap}
In the bootstrap phase the goal is to create a working foundation of the project which will enable
further concurrent development of the planned features. This phase consists primarily of low-level
design of the critical systems and modules, as well as early development of the blocking parts of
the program.

A hard time estimate for this phase is 1 week with no planned overruns.

\subsection{Core Development}
Core development is the largest phase of the project optimistically estimated to last 9 weeks. After
that time the project enters the finalisation phase regardless of whether an overrun occurs.

The main focus is on developing the core set of features of the application as those have the
highest priority and are essential to proper functioning and experience of CODENAME. In the unlikely
case that all core features are completed before the end of this phase, work on extension features
may start.

\subsection{Finalisation}
The finalisation phase is the last phase of the project. This phase serves primarily as a buffer for
an overrun in the core development phase so that all core features might be completed and polished.
Any extension features successfully developed before the end of this phase will be merged. The
planned duration of this phase is 1 week with a possible extension to 2 weeks.

\section{Architecture Overview}
CODENAME consists of several interoperating modules: model, renderer, executor, persistence, user
interface.

DIAGRAM HERE

\subsection{Model}
Model is the core module responsible for in-memory storage of the logical and structural
representation of the circuits and the related functionality. The Core Representation (CR)
encompasses gates, components, IO connectors, wires.

Gates are builtin atomic components which perform boolean operations on inputs. The only necessary
gate is NAND, however, for the convenience of a user and performance all basic operations are
included (NOT, AND, NAND, OR, NOR, XOR, XNOR). Additionally, SR Latch and D Latch may be considered
for inclusion in this set, as those are elemental memory units.

Components are groups of smaller components with defined IO acting as a single building block.

IO connectors are used to transfer into and out of components. Types of IO connectors are essential
to proper UX as errors resulting from mismatched types of IO connectors, such as input-input
connections or 8 bit to 1 bit connection, will be difficult to locate by the end user. No debugging
functionality is planned, hence minimising possible points of failure is crucial.

Wires are an explicit representation of the connections between components. Wires may have junctions
that split the wire into multiple lines transferring the same value. There is no distinction between
different bit width buses to simplify construction of more complex circuits.

The CR, for the purposes of execution, may be lowered to a performance oriented representation (PR)
which is built on-demand before execution and is structurally immutable.

\subsection{Renderer}
The renderer is fed geometry and color/texture data, and draws it on screen. The internal design is
entirely dependent on the graphics API, however, the interface of this module should be
straightforward in the way that it should be possible to issue individual draws efficiently. It is
an independent module that does not have any strict ties to any other modules.

\subsection{Executor}
Executor simulates the the circuit with a set of given inputs calculating the outputs of all gates
and subcomponents, and propagating them throughout the system. Executor supports execution triggered
by input change and continuous or step clock-based execution to simulate working of a real circuit.

\subsection{Persistence}
Storage of created designs is essential, as recreating them from scratch each time would be
deterring to say the least. Persistence must be capable of serialising the model to store it
externally and reconstruct the model from serialised data.

\subsection{User Interface}
User Interaface (UI) is the primary method of interaction with the program. UI exposes interactivity
points that manipulate the state of the program and handles user input, possible actions being:
\begin{itemlist}
\item dragging view,
\item creating a gate
\item moving a gate,
\item creating a wire between two IO connectors,
\item box-selecting gates,
\item grouping gates into a larger component,
\item starting simulation,
\item importing/exporting the design.
\end{itemlist}
UI should expose 2 views - main view of the circuit and component editor view. The component editor
is used to edit components, that is modify their internal structure and modify IO. In the case that
a component has its IO edited, an old version of the component should be maintained until all
occurences of it are replaced. UI should warn the user that instances of such a component must be
updated, ideally providing ways to navigate to such locations, and prevent execution until it is
done.

On top of that, UI prepares geometry data of the whole user facing side of the application that is
to be passed to the renderer and initiates rendering of a frame.

\subsection{Resolving Dependencies}
From the above we may conclude that the model is a critical component blocking other modules except
renderer and the design must be established early on to further development. There is a blocking
dependency of UI on the renderer as it is impossible to display any results on-screen without a
working renderer. This makes the model and the renderer the highest priority goals. A simple
bootstrap renderer should be developed early on to unblock other modules. An improved version would
then be developed incrementally and concurrently with other modules.

\section{Technical Evaluation}
The core technologies used in the project are:
\begin{itemlist}
\item C++
\item OpenGL
\item GLSL
\end{itemlist}

Additionally, the following libraries are evaluated:
\begin{itemlist}
\item anton{\_}core
\item GLFW
\item Dear Imgui
\end{itemlist}

\subsection{Evaluation of C++}
C++ is a widespread programming language with a large community of experts and a standard committee.
The language has been in use for over 30 years, especially favoured by the game development industry
for its performance, interoperability as well as a range of frameworks and libraries. It is safe to
assume that the compilers of this language have been thoroughly tested for our use cases.

The project mandates use of a language that has a high interoperability with graphics APIs. C++
fulfills this role is it allows for low-level manipulation of memory, but at the same time provides
facilities for high-level abstractions. The complexity of the language is overwhelming, however, the
commonly used subset of the language is well known and should not pose any risk in terms of possible
target overruns. Using other less known languages, such as Rust, would require an initial learning
period taking away from the development and would greatly contribute to stalls due to lack of
essential knowledge furthering overruns.

The version used is C++20 with the exception that only the C++17 feature set and a subset of C++20
features is allowed. This is due to the fact that many of the additions in C++20 have a major
overhead in terms of compilation speed.

\subsection{Evaluation of OpenGL}
OpenGL (GL) is an industry standard graphics API in use for over 30 years. The 4.5 version of GL
released in 2016 has been widely adopted with major GPU retailers adding support for this version to
their drivers within 2 years of the release. Over 5 years of continuous support proves its stability
and reliability.

Although version 4.0 and especially 3.3 have a much longer tenure and larger adoption, the newly
added features in versions 4.1-4.5 are extremely convenient and provide tools for the programmers,
such as Debug Output (dbgout) which outputs robust debug diagnostics on various errors easing
development and debugging.

GL provides two profiles: core and compatibility. The compatibility profile is designed for
upgrading existing software to newer versions of GL while maintaining the existing codebase. We do
not have any legacy code, hence compatibility profile is not required and we may proceed with core.

An alternative to GL is Vulkan (Vk) which is a newer API targeted at ultrahigh performance graphics
programs. Primarily used in game development, Vk provides a more robust API allowing a fine control
over the hardware. Vk has several disadvantages over GL for our purposes:
\begin{itemlist}
\item Vk is difficult. The concepts innate to Vk are difficult for beginners as it exposes the inner
  workings of the GPU drivers to a greater extent, primarily to allow for the fine control. An
  example is synchronisation which is done automatically in GL, but in Vk requires a considerable
  amount of work.
\item Vk is verbose. Compared to GL, the exact same boilerplate program requires several times more
  code to set up. Further development of the program in Vk would require proportionally more code
  hindering speed of development.
\end{itemlist}

CODENAME is not a graphics intensive program, hence the benefits offered by Vk are minimal while the
disadvantages may significantly slow down the development and cause target overruns.

Other mainstream options, such as Metal, DirectX, are not viable due to being locked to specific
platforms which are not out target platforms.

\subsection{Evaluation of GLSL}
GLSL, OpenGL Shading Language, is a shading language for creating GPU shading programs (shaders)
designed to work directly with GL. This compatibility naturally translates to a simpler workflow as
other options require cross-compilation to intermediate formats to be able to load them. There are
no performance implications of choosing GLSL over other languages.

GLSL allows the users to choose a version of the compiler that will be used at load time. To remain
consistent with the version of GL, we use version '450 core'.

\subsection{Evaluation of anton{\_}core}
TODO

\subsection{Evaluation of GLFW}
TODO

\subsection{Evaluation of Dear Imgui}
TODO

\end{document}
