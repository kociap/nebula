\documentclass[12pt]{article}

\usepackage[a4paper, margin=2cm]{geometry}
\usepackage{polski}
\usepackage[utf8]{inputenc}
\usepackage{graphicx}
\usepackage{enumitem}
\usepackage{algpseudocode}
\usepackage{algorithm}

\title{Nebula -- Persistence Documentation}
\author{Michał Biszczanik, Miłosz Chojecki}
\date{}

\begin{document}

\maketitle

\section*{Introduction}

Storage of created designs is essential, as recreating them from scratch each
time would be deterring to say the least. Persistence is capable of
serialising the model to store it externally and reconstructing the model from
serialised data.

\section*{How it works}

User specifies input and output files as command-line arguments: \\
\verb|./nebula [-l in_file.neb] [-s out_file.neb]| \\
If no input file is specified, Nebula loads a sample project consisting of two logic gates. \\
If no output file is specified, the project is discarded when exiting Nebula.

\section*{Structure of a .neb file}

An example \verb|.neb| file looks like this:
\begin{verbatim}
gate 0.6 0.5 -2.76016 2.21604 1
gate 0.6 0.5 -0.978603 1.93038 0
endgates
port -2.76016 2.34104 0
port -2.76016 2.59104 0
port -2.16016 2.46604 1
port -0.978603 2.05538 0
port -0.978603 2.30538 0
port -0.3786 2.18038 1
endports
connection -2.16016 2.46604 -0.978603 2.05538
connection -2.16016 2.46604 -0.978603 2.30538
connection -0.978603 2.05538 -2.16016 2.46604
connection -0.978603 2.30538 -2.16016 2.46604
endconnections
\end{verbatim}

The file consists of three sections: gates, ports and connections. \\
Each gate is stored as follows: \\
\verb|gate <x_size> <y_size> <x_coord> <y_coord> <gate_type>| \\
Each port is stored as follows: \\
\verb|port <x_coord> <y_coord> <type>| \\
Each connection is stored as follows: \\
\verb|connection <first_x_coord> <first_y_coord> <second_x_coord> <second_y_coord>|.

\section*{Class diagram of persistence module}

\includegraphics[]{class-persistence.png}

\section*{Swimlane diagram of persistence module}

\includegraphics[scale=0.85]{swimlane-persistence1.png}

\end{document}

