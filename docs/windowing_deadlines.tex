\documentclass{article}

\title{Windowing - Nebula}
\author{Sebastian W.}

\usepackage{amssymb}
\begin{document}

\maketitle

This module is intended to manage the application window and support the first layer of user interaction. The main purpose of this component, called "windowing", is to convert user input into a manageable format, present the main application window, handle interaction with that window, and provide context for rendering. We will also include a week of time in the program to learn libraries such as GLFW in C++. Below are the individual planned implementation phases. The deadlines may change slightly depending on the pace of creation of related program modules (mainly UI).
\begin{itemize}

\item[$\checkmark$] \textbf{Week 1: Planning and Design}
    \begin{itemize}
        \item Defining design details and functionality
        \item Creating a work schedule for 10 weeks
        \item Preparing the programming environment
        \item Start learning the GLFW library
    \end{itemize}

\item[$\times$] \textbf{Week 2: GLFW Advanced Learning}
    \begin{itemize}
        \item GLFW library continuing education
        \item Creating your first window using GLFW
        \item Handling user events in the window
    \end{itemize}

\item[$\times$] \textbf{Week 3: Implement User Input Handling}
    \begin{itemize}
        \item Process and handle user input (keyboard, mouse)
        \item Convert input data into a manageable format
    \end{itemize}
    
    \item[$\times$] \textbf{Week 4-5: Create the Main Application Window}
    \begin{itemize}
        \item Design and create the main application window
        \item Implement interactions with the window (dragging, resizing, etc.)
    \end{itemize}
    
    \item[$\times$] \textbf{Week 6: Provide Rendering Context}
    \begin{itemize}
        \item Create and manage the rendering context
        \item Implement basic rendering in the window
    \end{itemize}
    
    \item[$\times$] \textbf{Week 7-9: Testing and Refining the project}
    \begin{itemize}
        \item Test the program for errors and inefficiencies
        \item Optimize code and resources
        \item Adding more than basic functionalities that will facilitate user interactions
    \end{itemize}
\end{itemize}

\end{document}
